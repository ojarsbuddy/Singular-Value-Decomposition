\documentclass{article}
\usepackage{amsmath,amssymb,physics,xfrac,nicefrac}
\usepackage{standalone}
\usepackage[left=1.5cm,top=1.5cm,right=1.5cm,bottom=1.5cm]{geometry}
\usepackage[indentfirst=true]{quoting}
\usepackage{xcolor, soul}
\usepackage{enumitem}
\definecolor{HighlightGray}{cmyk}{0,0,0,0.07}
\sethlcolor{HighlightGray}
 
\title{Singular Value Decomposition}
\author{John Bortins}
 
\begin{document}

\maketitle{}

\begin{align*}
    A=      & \begin{pmatrix}
        3 & 4 \\0&5\\
    \end{pmatrix}                                                                                                                \\
    A^T=    & \begin{pmatrix}
        3 & 0 \\4&5\\
    \end{pmatrix}                                                                                                                \\
    A^T A=  & \begin{pmatrix}
        3 & 0 \\4&5\\
    \end{pmatrix}\begin{pmatrix}
        3 & 4 \\0&5\\
    \end{pmatrix}=\begin{pmatrix}
        9 & 12 \\12&41\\
    \end{pmatrix}                                                             \\
    A^T Av= & \begin{pmatrix}
        9 & 12 \\12&41\\
    \end{pmatrix}\begin{pmatrix}
        v_1 \\v_2\\
    \end{pmatrix}=\begin{pmatrix}
        9v_1 + 12v_2 \\12v_1 +41v_2
    \end{pmatrix}=\lambda\begin{pmatrix}
        v_1 \\v_2
    \end{pmatrix}=\begin{pmatrix}
        \lambda v_1 \\\lambda v_2
    \end{pmatrix} \\
\end{align*}
Find the eigenvalues $\lambda$ by solving the equations.
\begin{align*}
    \begin{pmatrix}
        (9-\lambda) & 12 \\12 &(41-\lambda)
    \end{pmatrix}\begin{pmatrix}
        v_1 \\v_2
    \end{pmatrix} & =  \begin{pmatrix}
        0 \\0
    \end{pmatrix}    \\
    (9-\lambda)(41-\lambda)-144                          & =                           0    \\
    \lambda^2 -50\lambda+225                             & =                              0 \\
    (\lambda-5)(\lambda-45)                              & =0
\end{align*}
\begin{align*}
    \qfor*\lambda=5\qquad\begin{pmatrix}
        4 & 12 \\12 &36
    \end{pmatrix}\begin{pmatrix}
        v_1 \\v_2
    \end{pmatrix}  & =  \begin{pmatrix}
        0 \\0
    \end{pmatrix}\implies v_1+3v_2=0 \\
    \qq*{the corresponding eigenvectors are nonzero multiples of}              & \begin{pmatrix}
        3 \\ -1
    \end{pmatrix}                       \\                                                                                              \\
    \qfor*\lambda=45\qquad\begin{pmatrix}
        -36 & 12 \\12 &-4
    \end{pmatrix}\begin{pmatrix}
        v_1 \\v_2
    \end{pmatrix} & =  \begin{pmatrix}
        0 \\0
    \end{pmatrix}\implies 3v_1-v_2=0 \\
    \qq*{the corresponding eigenvectors are nonzero multiples of}              & \begin{pmatrix}
        1 \\ 3
    \end{pmatrix}                       \\                                                                                              \\
\end{align*}
\end{document}