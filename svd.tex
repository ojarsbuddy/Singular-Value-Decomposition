\documentclass{article}
\usepackage{amsmath,amssymb,physics,xfrac,nicefrac}
\usepackage{standalone}
\usepackage[left=1.5cm,top=1.5cm,right=1.5cm,bottom=1.5cm]{geometry}
\usepackage[indentfirst=true]{quoting}
\usepackage{xcolor, soul}
\usepackage{enumitem}
\definecolor{HighlightGray}{cmyk}{0,0,0,0.07}
\sethlcolor{HighlightGray}
 
\title{Singular Value Decomposition}
\author{John Bortins}
 
\begin{document}

\maketitle{}

\section{Introduction}
Start with some matrix $A$ and its transpose $A^T$. Both $A^T A$ and $AA^T$ are square, symmetric, and  nonnegative definite.

We factor any $A=U\Sigma V^T$ with $U^T U=I$, $\Sigma$ diagonal and $V^T V=I$. Now think of $A^T A=(V\Sigma^T U^T)(U\Sigma V^T)=V\Sigma^T \Sigma V^T$ in  order to discover eigenvalues $\lambda$ of $A^T A$. Note that $A$ has two sets of singular vectors (the eigenvectors of $A^T A$ and$AA^T$).
\begin{align*}
    A      & =U\Sigma V^T\qand A^T=V\Sigma^T U^T \\
    A^T A  & =(V\Sigma^T U^T)(U\Sigma V^T)       \\
           & =V\Sigma^T \Sigma V^T               \\
           & =V\Lambda V^T                       \\
    A^T AV & =V\Lambda
\end{align*}

Consider $A^T AV=V\Lambda$ and choose some column vector $v_i$ of $V$, then $A^T Av_i=\lambda_i v_i$ can pick off the eigenvalues. Both $A^T A$ and $AA^T$ have the same set of eigenvalues, which means there is one set of singular values, where $\lambda_i=\sigma_i^2$.

To find $u_i$ in the column space of $U$, note that $AV=U\Sigma$ and $Av_i=u_i\sigma_i$.

Recall that the general quintic cannot be solved by radicals per Abel and Galois supplied the conditions by which higher order ($>4$) equations can be so solved.

\section{An Example }
\begin{align*}
    A=      & \begin{pmatrix}
        3 & 4 \\0&5\\
    \end{pmatrix}                                                                                                                 \\
    A^T=    & \begin{pmatrix}
        3 & 0 \\4&5\\
    \end{pmatrix}                                                                                                                 \\
    A^T A=  & \begin{pmatrix}
        3 & 0 \\4&5\\
    \end{pmatrix}\begin{pmatrix}
        3 & 4 \\0&5\\
    \end{pmatrix}=\begin{pmatrix}
        9 & 12 \\12&41\\
    \end{pmatrix}                                                              \\
    A^T Av= & \begin{pmatrix}
        9 & 12 \\12&41\\
    \end{pmatrix}\begin{pmatrix}
        v_x \\v_y\\
    \end{pmatrix}=\begin{pmatrix}
        9v_x + 12v_y \\12v_x +41v_y
    \end{pmatrix}=\lambda\begin{pmatrix}
        v_x \\v_y
    \end{pmatrix}=\begin{pmatrix}
        \lambda v_x \\\lambda v_y
    \end{pmatrix} \\
\end{align*}
Find the eigenvalues $\lambda$ by solving the equations.
\begin{align*}
    (A^T A-\lambda_i I)v_i=\begin{pmatrix}
        (9-\lambda) & 12 \\12 &(41-\lambda)
    \end{pmatrix}v_i & =  \begin{pmatrix}
        0 \\0
    \end{pmatrix}    \\
    (9-\lambda)(41-\lambda)-144                          & =                           0    \\
    \lambda^2 -50\lambda+225                             & =                              0 \\
    (\lambda-45)(\lambda-5)                              & =0
\end{align*}
\begin{align*}
    \qq*{For}\lambda_1=45\qquad\begin{pmatrix}
        -36 & 12 \\12 &-4
    \end{pmatrix}\begin{pmatrix}
        v_x \\v_y
    \end{pmatrix} & =  \begin{pmatrix}
        0 \\0
    \end{pmatrix}\implies 3v_x-v_y=0 \\
    \qq*{the corresponding eigenvectors are nonzero multiples of}                   & \begin{pmatrix}
        1 \\ 3
    \end{pmatrix}                       \\
    \qq*{and corresponding unit vector}v_1=\frac{1}{\sqrt{10}}                      & \begin{pmatrix}1\\3\end{pmatrix}                       \\
    \qq*{For}\lambda_2=5\qquad\begin{pmatrix}4 & 12 \\12 &36\end{pmatrix}\begin{pmatrix}v_x \\v_y
    \end{pmatrix}  & =  \begin{pmatrix}0 \\0
    \end{pmatrix}\implies v_x+3v_y=0 \\
    \qq*{the corresponding eigenvectors are nonzero multiples of}                   & \begin{pmatrix}
        3 \\ -1\end{pmatrix}                       \\                                                                               \qq*{and corresponding unit vector}v_2=\frac{1}{\sqrt{10}}                   & \begin{pmatrix}
        3 \\ -1
    \end{pmatrix}                       \\
    \\
\end{align*}
\begin{align*}
    \Sigma & =\begin{pmatrix}
        \sqrt{45} & 0 \\0&\sqrt{5}
    \end{pmatrix}
\end{align*}
\begin{align*}
    V =V^T & =\frac{1}{\sqrt{10}}\begin{pmatrix}1 & 3 \\3 & -1\\\end{pmatrix}                                                                                     \\
    V^T V  & =\frac{1}{\sqrt{10}}\begin{pmatrix}1 & 3 \\3 & -1\\\end{pmatrix}\frac{1}{\sqrt{10}}\begin{pmatrix}1 & 3 \\3 & -1\\\end{pmatrix}=\frac{1}{10}\begin{pmatrix}10 & 0 \\0&10\\\end{pmatrix} \\
    V^T V  & =I
\end{align*}
\begin{align*}
    u_1=\frac{Av_1}{\sigma_1} & =\frac{1}{\sqrt{45}}\begin{pmatrix}3&4\\0&5\\\end{pmatrix}\frac{1}{\sqrt{10}}\begin{pmatrix}1\\3\end{pmatrix}=\frac{1}{\sqrt{450}}\begin{pmatrix}15\\15\end{pmatrix}=\frac{15}{\sqrt{2(225)}}\begin{pmatrix}1\\1\end{pmatrix} \\
    u_1                       & =\frac{1}{\sqrt{2}}\begin{pmatrix}1\\1\\\end{pmatrix}                                                                                                                                                 \\
    u_2=\frac{Av_2}{\sigma_2} & =\frac{1}{\sqrt{5}}\begin{pmatrix}3&4\\0&5\\\end{pmatrix}\frac{1}{\sqrt{10}}\begin{pmatrix}3\\-1\end{pmatrix}=\frac{1}{\sqrt{50}}\begin{pmatrix}5\\-5\end{pmatrix}=\frac{5}{\sqrt{2(25)}}\begin{pmatrix}1\\-1\end{pmatrix}     \\
    u_2                       & =\frac{1}{\sqrt{2}}\begin{pmatrix}1\\-1\\\end{pmatrix}                                                                                                                                                 \\
\end{align*}
\end{document}